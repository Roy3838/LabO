
\begin{document}

\preprint{APS/123-QED}

\title{Pr\'{a}ctica \#5 \\ Interferómetro de Michelson}% Force line breaks with \\
\thanks{A footnote to the article title}%

\author{Luis Rodrigo Medina Murúa}
 \altaffiliation[Also at ]{Departamento de física, Instituto Tecnológico de Monterrey}%Lines break automatically or can be forced with \\
 \email{A01283783@tec.mx}

\date{\today}% It is always \today, today,
             %  but any date may be explicitly specified

\begin{abstract}
El objetivo de la práctica es familiarizarse con el interferómetro de Mach-Zehnder, que es uno de los interferómetros más útiles por su versatilidad. 
\begin{description}
\item[Objetivo:]
Realizar un interferómetro de Mach-Zehnder.
\item[Objetivo:]
Medir la longitud de coherencia de una fuente de luz.
\item[Objetivo:]
Familiarizarse con los detalles de la alineación del interferómetro.
\end{description}
\end{abstract}

%\keywords{Suggested keywords}%Use showkeys class option if keyword
                              %display desired
\maketitle

%\tableofcontents




\section{Descripción}

El interferómetro de Mach-Zehnder es un dispositivo que consiste en dos divisores de haz y dos espejos que forman dos caminos ópticos separados para los haces de luz provenientes de una misma fuente. Los haces se recombinan en un segundo divisor de haz y se observa la interferencia en una pantalla o un detector. El interferómetro se puede usar para medir pequeños cambios de fase inducidos por una muestra colocada en uno de los caminos o por una diferencia de longitud entre los caminos.

Un beneficio del interferómetro de Mach-Zehnder en comparación con el de Michelson es que es más configurable y flexible. Por ejemplo, se puede colocar una muestra en uno de los caminos ópticos sin afectar al otro. Además, el interferómetro de Mach-Zehnder produce dos imágenes con una diferencia de fase de $\pi$, lo que lo hace más útil en ciertas circunstancias que el interferómetro de Michelson que produce una sola imagen. Sin embargo, el interferómetro de Mach-Zehnder también tiene algunas desventajas, como la necesidad de igualar cuidadosamente los dos caminos ópticos si la fuente tiene una baja longitud de coherencia.

El interferómetro de Mach-Zehnder tiene diversas aplicaciones en los campos de la aerodinámica, la física de plasma y la transferencia de calor, ya que puede medir cambios en la presión, la densidad y la temperatura en gases mediante el análisis del patrón de interferencia\cite{wikiwand} \cite{problemasresueltos} También se utiliza para realizar experimentos sobre las propiedades cuánticas de la luz, como el entrelazamiento y la teleportación cuántica.


\section{Marco Teórico}

Marco Te ́orico. Abarca los fundamentos te ́oricos, principios f ́ısicos y matem ́aticos, ecua-
ciones y f ́ormulas relacionadas con el experimento y fen ́omenos a estudiar






